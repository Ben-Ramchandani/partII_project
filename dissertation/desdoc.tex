\documentclass[oneside]{article}

\textwidth=460pt
\textheight=680pt
\marginparwidth=51pt
\hoffset=-1in
\voffset=-1in

\setlength{\parindent}{0pt}

\usepackage{graphicx}
\usepackage{enumerate}

\title{
{\Large Computer Science Tripos -- Part II}\\
\vspace{1em}
Incentivising distributed file storage using blockchain contracts.\\
Design documentation.}
\author{Ben Ramchandani \\ bjr39@cam.ac.uk \\ St. Catharine's College}

\begin{document}
\maketitle

\textbf{Supervisor}: John Knox\\
\textbf{Director of Studies}: Dr Sergei Taraskin\\
\textbf{Project Overseers}: I. Wassell, L. Paulson\\
\textbf{Originator}: Ben Ramchandani

\section{File disimentation}

The file will be split into small blocks of size 1024kb.
It is one of these blocks that will be selected for the proof of retrieveability.

\section{Merkle tree}

For the purposes of Merkle tree construction the file will be zero padded up such that the number of blocks required is a power of two.
The number of leaves in the Merkle tree will therefore be a power of 2, and the size of the proof will therefore be the same regardless of which block of the file is selected.

I will write a Merkle tree implementation in Java, with the following functionality:

Given a file, it will zero pad the file as described above, before splitting it into blocks and building a Merkle tree.
It will then output $H$, the root hash of this tree.

\section{Proof generation}

Another Java program.
This one takes a file and a number $i$ of one of the blocks in the file (not the padding blocks) and outputs a proof of the form
\begin{verbatim}[FILE BLOCK 1024 BYTES #######] || [SHA256] || [SHA256] ...\end{verbatim}
where there are $\log_2(\texttt{num\_blocks})$ SHA256 hashes.
This proof will therefore have a size of $1024 + \log_2(\texttt{num\_blocks}) \times 32$ bytes.
Note that no bounding markers are needed due to the fixed size.

\section{Contract}

The contract needs to store the following values:
\begin{itemize}
\item $T_0$, the number of the blockchain block the contract was initialised on.
\item $N$, the number of blockchain blocks the contract should wait.
\item $m$, the number of blocks in the file (before zero-padding).
\item $H$, the Merkle tree root hash.
\item $X$, the amount of ether to pay out.
\end{itemize}

When the contract is called with a proof it should calculate an $0 \leq i < m$ from the hash of block $T_0 + N$ then check the following statements.
If any of them are true the contract should `throw`, aborting the transaction.
\begin{itemize}
\item The contract has already paid out.
\item $i \geq T_0 + N$. 
\item $i < T_0 + N + 256$ (if this is the case the block hash is no longer available).
\item Proof is invalid.
\end{itemize}


\section{Proof construction algorithm}

\begin{verbatim}
input int i; // Block number
input int m; // Total number of blocks in file (unpadded)
input Node root; // Merkle tree root
input MTree root;
let M = 2^(ceil(log2(m)));
let j = 1024 + log2(M) * 32;
let p = byte[j];
x = M / 2;
node = root;
while(x > 1) {
    if (i < x) {
        p[j - 32, j] = SHA256(root.right);
        node = node.left;
        x = x / 2;
    } else {
        p[j - 32, j] = SHA256(root.left);
        node = node.right;
        x = x + x / 2;
    }
    j -= 32
}
assert(j == 1024);
assert(node is leaf);
p[0, 1024] = SHA256(node);
return p;
\end{verbatim}

\section{Proof validation algorithm}

\begin{verbatim}
input int m; // Total number of blocks in file (unpadded)
input int H; // Merkle tree root hash.
input byte[] proof; // The proof.
let i = blockhash(T_0 + N) % m; // i Is the block required for the proof.
let h = SHA256(proof[0, 1024]);
let M = 2^(ceil(log2(m)));
let j = 1024;
let c = 0;
while(c < M) {
    // Is the cth bit of i 0? Reconstruct the tree branching bottom up. 
    if(i & (1 << c) == 0) { // Assuming big endian, need to check what the EVM uses.
        h = SHA256(h | proof[j, 256]); // here | denotes concatenation.
    } else {
        h = SHA256(proof[j, 256] | h);
    }
    c += 1;
    j += 256;
}
return h == H;
\end{verbatim}

\end{document}
